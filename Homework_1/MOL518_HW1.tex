% MOL518 Homework 1
% (Edit due date, points, and any course policy text as needed.)

\documentclass[11pt]{article}

\usepackage[margin=1in]{geometry}
\usepackage{amsmath, amssymb}
\usepackage{graphicx}
\usepackage{hyperref}
\usepackage{enumitem}
\usepackage{float}
\usepackage{booktabs}
\usepackage{xcolor}
\usepackage{listings}

\hypersetup{colorlinks=true, linkcolor=blue, urlcolor=blue, citecolor=blue}

% Simple code listing style
\lstset{
  basicstyle=\ttfamily\small,
  frame=single,
  breaklines=true,
  showstringspaces=false,
  keywordstyle=\color{blue},
  commentstyle=\color{gray},
  stringstyle=\color{teal}
}

\newcommand{\course}{MOL518 Spring 2026}
\newcommand{\hwnum}{Homework 1}
\newcommand{\duedate}{\textbf{Due: Friday February 6, 2026}}

\begin{document}

\begin{center}
{\Large \course}\\[2pt]
{\large \hwnum}\\[6pt]
\duedate\\[10pt]
\end{center}


\begin{itemize}
  \item For this assignment, you will generate and then submit a jupyter notebook on canvas.
  \item You may use generative AI tools as per the syllabus unless otherwise stated in the problem.
\end{itemize}

\textit{First, make a new Jupyter notebook named \texttt{MOL518\_HW1\_<YourLastName>.ipynb}. All problems will be answered in this notebook. For each problem, start with a Markdown cell using Problem XX as the title, i.e. ``\# Problem 1''.}

\section*{Problem 1: Practice with Markdown}

Figure~\ref{fig:stryerpreface} is the preface page for Lubert Stryer's classic textbook \emph{Biochemistry}. Recreate this page in a Markdown cell in your notebook. The image at the bottom of the page is provided as \texttt{StryerPreface\_liverketone.png}. {\it Do not use AI for this problem.}

\begin{figure}[H]
\centering
\includegraphics[width=0.6\textwidth]{StryerPreface.png}
\caption{Stryer preface page.}
\label{fig:stryerpreface}
\end{figure}

\section*{Problem 3: NumPy Arrays (Indexing, Slicing, Shape)}
(25 points)

\begin{enumerate}[label=\alph*)]
  \item Create a 1D NumPy array containing the odd integers from 1 to 21 (inclusive).
  \item Print every 3rd element of that array (starting from the first element of the array).
  \item How many elements remain if you remove every 3rd element? Show your reasoning and/or code.
  \item Create the following 2D array as a NumPy array:

\begin{center}
\begin{tabular}{ccccc}
\toprule
0 & 0 & 1 & 0 & 0\\
0 & 2 & 0 & 2 & 0\\
3 & 0 & 0 & 0 & 3\\
0 & 2 & 0 & 2 & 0\\
0 & 0 & 1 & 0 & 0\\
\bottomrule
\end{tabular}
\end{center}

  \item Print the 3rd row (be careful about Python indexing).
  \item Compute the total number of elements in the array using its \texttt{.shape}.
  \item Replace the last row with \texttt{[99, 99, 99, 99, 99]} and print the modified array.
\end{enumerate}

\section*{Problem 4: Loading and Inspecting Data (CSV)}
(25 points)

In Lecture 2 you loaded a growth curve CSV file into a NumPy array.

\begin{enumerate}[label=\alph*)]
  \item Load \texttt{Lecture\_2/data/growth\_curve1.csv} using \texttt{np.loadtxt(..., delimiter=",")}. Include the code you used.
  \item Print the shape of the loaded array.
  \item Print the first 5 rows and the last row.
  \item Extract the time column and OD column into 1D arrays called \texttt{time} and \texttt{od}.
  \item Compute the total duration of the experiment in hours.
\end{enumerate}

\section*{Problem 5: Saving Derived Data}
(15 points)

\begin{enumerate}[label=\alph*)]
  \item Create a normalized OD array \texttt{od\_norm} by dividing all OD values by the first OD measurement.
  \item Create a 2D array with time in hours and \texttt{od\_norm} as columns.
  \item Save the result as a new CSV file (do \emph{not} overwrite the original). Name it \texttt{growth\_curve1\_hr\_norm.csv} and place it alongside the original in the same \texttt{data/} directory.
\end{enumerate}

\section*{Optional (Extra Credit): Two Files, Lists, and Change in OD}
(+5 points)

Load both \texttt{growth\_curve1.csv} and \texttt{growth\_curve2.csv} using a Python list of filenames. For each dataset, compute the change in OD from the first to last time point. Print the two values.

\vspace{8pt}
\noindent\textbf{Checklist before submitting:}
\begin{itemize}[leftmargin=*]
  \item PDF compiles without errors.
  \item All questions answered and labeled.
  \item Code included where requested.
  \item Output/units reported where relevant.
\end{itemize}

\end{document}
