\documentclass[11pt]{article}

\usepackage[margin=1in]{geometry}
\usepackage{amsmath, amssymb}
\usepackage{graphicx}
\usepackage{hyperref}
\usepackage{enumitem}
\usepackage{float}
\usepackage{booktabs}
\usepackage{xcolor}
\usepackage{listings}

\hypersetup{colorlinks=true, linkcolor=blue, urlcolor=blue, citecolor=blue}

% Simple code listing style
\lstset{
  basicstyle=\ttfamily\small,
  frame=single,
  breaklines=true,
  showstringspaces=false,
  keywordstyle=\color{blue},
  commentstyle=\color{gray},
  stringstyle=\color{teal}
}

\newcommand{\course}{MOL518 Spring 2026}
\newcommand{\hwnum}{Homework 2}
\newcommand{\duedate}{\textbf{Due: Friday February 13, 2026}}

\begin{document}

\begin{center}
{\Large \course}\\[2pt]
{\large \hwnum}\\[6pt]
\duedate\\[10pt]
\end{center}


For this assignment, you will generate and then submit a Jupyter notebook on Canvas. You may use generative AI tools as per the syllabus unless otherwise stated in the problem.

\vskip 4pt

\noindent \textit{First, make a new Jupyter notebook named \texttt{MOL518\_HW2\_<YourLastName>.ipynb}. All problems will be answered in this notebook. For each problem, start with a Markdown cell using Problem XX as the title, i.e. ``\# Problem 1''.}


\newpage

\section*{Problem 1: Looping Through Subplots}

Use the file \texttt{ecoli\_drugs.csv}. This file has time in the first column and six drug growth curves in the next six columns. {\it Do not use AI to generate code for this problem. You may use AI to debug if you get an error.}

\begin{enumerate}[label=\alph*)]
  \item Load the CSV file using NumPy (skip the header row).
  \item Create a list of the six OD arrays in the order: Rifampicin, Novabiocin, Trimethoprim, Chloramphenicol, Ampicillin, Gentamycin.
  \item Create a matching list of drug name strings.
  \item Use \texttt{plt.subplots} to make a 3x2 grid. Fill the panels using \textbf{for loops only} (no manual \texttt{axes[0,0]} indexing). Each panel should include a title and axis labels.
  \item Save the figure as \texttt{drug\_growth\_curves\_hw2.png} and \texttt{drug\_growth\_curves\_hw2.pdf}.
\end{enumerate}

\section*{Problem 2: Mystery Scatter + Threshold Count}

Use the file \texttt{Homework2/HW2\_data/mystery.csv}. {\it Do not use AI to generate code for this problem. You may use AI to debug if you get an error.}

\begin{enumerate}[label=\alph*)]
  \item Load the data into two 1D arrays called \texttt{x} and \texttt{y}.
  \item Make a scatter plot of \texttt{y} vs \texttt{x}. Label both axes.
  \item Using a \texttt{for} loop and an \texttt{if} statement, count how many points have \texttt{y > 25}. Store the count as \texttt{n\_above}.
  \item Save a tiny CSV file called \texttt{mystery\_summary.csv} with two columns: \texttt{threshold} and \texttt{n\_above}.
\end{enumerate}

\section*{Problem 3: File System Scavenger (Microscopy Images)}

In \texttt{Homework2/HW2\_data/images} there are 20 microscopy-style \texttt{.tif} files with scientific filenames. {\it Do not use AI to generate code for this problem. You may use AI to debug if you get an error.}

\begin{enumerate}[label=\alph*)]
  \item Use \texttt{pathlib.Path} to list all \texttt{.tif} files in this folder and print the total count.
  \item Print the names of the first 5 files (alphabetical order).
  \item For each filename, split the stem on underscores (\texttt{\_}) and extract the cell line (the second field, e.g. \texttt{HeLa} in \texttt{2026-02-05\_HeLa\_DAPI\_ctrl\_FOV01.tif}).
  \item Use a dictionary to count how many images belong to each cell line, then print the dictionary.
\end{enumerate}

\section*{Problem 4: Bimodal Expression and Bin Size}

Use the file \texttt{Homework2/HW2\_data/SOX2\_expression.csv} which contains one column of expression values. {\it Do not use AI to generate code for this problem. You may use AI to debug if you get an error.}

\begin{enumerate}[label=\alph*)]
  \item Load the expression values into a 1D array.
  \item Make a histogram with a small number of bins (e.g. 8). Title the plot and label axes.
  \item Create a 1x3 grid of histograms for bin counts \texttt{[8, 20, 40]}. Use \texttt{plt.subplots} and a \texttt{for} loop to fill the panels.
  \item In 2--3 sentences, explain how the choice of bin size changes what you perceive about the distribution.
\end{enumerate}

\section*{Problem 5: Replicate Growth Curves and the Mean}

Use the file \texttt{Homework2/HW2\_data/ecoli\_growth\_replicates.csv}. The first column is time (hours) and the next 10 columns are OD replicates. {\it Do not use AI to generate code for this problem. You may use AI to debug if you get an error.}

\begin{enumerate}[label=\alph*)]
  \item Load the CSV file into a 2D NumPy array. Extract \texttt{time} and a 2D array \texttt{od\_reps}.
  \item Plot all 10 replicate curves on the same axes in different colors.
  \item Compute the mean OD at each timepoint (average across replicates). Plot the mean as a thick black line on the same figure.
  \item Save a new CSV file called \texttt{ecoli\_growth\_mean.csv} with two columns: \texttt{time\_hr} and \texttt{od\_mean}.
\end{enumerate}


\end{document}
